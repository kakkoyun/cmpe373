\chapter{Implementation}
\section{Agents}

\subsection{Author}
\begin{lstlisting}
  public class Author implements ISDoS{
	private int id;
	private ArrayList<Discipline> disciplines = new ArrayList<Discipline>();
	private ArrayList<Paper> papers = new ArrayList<Paper>();
	private Community community;

	public Author() {}

	public Author(int id) {
		this.id = id;
		this.setCommunity(new Community());
	}

	public void step(Double pW, Paper paper) {
		this.papers.add(paper); // Add paper to author's written papers.
		paper.getAuthors().add(this); // Add author to paper's co-author list.
		paper.getSharedDisciplines().addAll(disciplines); // Diffusion of knowledge, pass collective knowledge through paper.
		
		// Get environmental containers.
		Context<ISDoS> context = ContextUtils.getContext(this);
		// Get co-authership network.
		Network<ISDoS> net = (Network<ISDoS>)context.getProjection("coAuthorship network");
		
		if(RandomHelper.nextDoubleFromTo(0, 1) < pW){
			System.out.println("Next walk"); // A tracker for debugging.
			
			double[] pdf = new double[net.getDegree(this)]; // Probability distribution function among neighbors of author in network.
			
			double totalWeight = 0; // Total weight of edges in network.
			for(RepastEdge<ISDoS> edge : net.getEdges()){
				totalWeight += edge.getWeight();
			}
			int index = 0;
			ArrayList<Object> neighbours = new ArrayList<Object>();
			for(RepastEdge<ISDoS> edge : net.getEdges(this)){
				double weightOfedge = edge.getWeight();
				double transitionProbability = weightOfedge / totalWeight; // Transition probability as explained above.
				pdf[index] = transitionProbability;
				index++;
				neighbours.add(edge.getTarget());
			}
			
			RandomHelper.createEmpiricalWalker(pdf, 0); // A random distribution according to given pdf.
			int indexOfneighbour = RandomHelper.getEmpiricalWalker().nextInt();
			Author nextAuthor = (Author)neighbours.get(indexOfneighbour); // Get next possible co-author candidate.
			nextAuthor.step(pW, paper); // Continue to walk through next candidate author.
			
		} else {
			paper.decideDiscipline(context); // Decide main discipline of paper.
			paper.updateDisciplinesOfAuthors(); // Update disciplines of co-authors.
			paper.updateCoAuthorNetwork(); // Update network according new knowledge.
		}
	}
\end{lstlisting}

\subsection{Paper}
\begin{lstlisting}
 public class Paper implements ISDoS{
	private Integer id;
	private ArrayList<Author> authors = new ArrayList<Author>();
	private Discipline discipline;
	private ArrayList<Discipline> unionOfSharedDisciplines = new ArrayList<Discipline>();
	
	public Paper(Integer id) {
		this.id = id;
	}

	public Paper(Integer id, ArrayList<Author> authors, Discipline discipline,
			ArrayList<Discipline> sharedDisciplines) {
		this.id = id;
		this.authors = authors;
		this.discipline = discipline;
		this.unionOfSharedDisciplines = sharedDisciplines;
	}

	public void decideDiscipline(Context<ISDoS> context){
		IndexedIterable<ISDoS> allDisciplines = context.getObjects(Discipline.class);
		int[] disciplineCount = new int[allDisciplines.size()];
		for(Discipline d : unionOfSharedDisciplines){
			disciplineCount[d.getId()] ++;
		}
		// For this stage only one discipline is set for a paper.
		int mostlySharedDisciplineID = 0;
		for(int i = 0; i < disciplineCount.length; i++){
			if(disciplineCount[i] > mostlySharedDisciplineID){
				mostlySharedDisciplineID = i;
			}
		}
		 setDiscipline((Discipline)allDisciplines.get(mostlySharedDisciplineID));
	}
	
	public void updateDisciplinesOfAuthors(){
		for(Author author : authors){
			if(!author.getDisciplines().contains(discipline)){
				author.getDisciplines().add(discipline);
			}
		}
	}
	
	public void updateCoAuthorNetwork(){
		Context<ISourceContext> context = ContextUtils.getContext(this);
		Network<ISDoS> net = (Network<ISDoS>)context.getProjection("coAuthorship network");
		System.out.println("Authors:" + authors.size());
		for(int i = 0; i < authors.size(); i++){
			for(int j = i+1; j < authors.size(); j++){
				System.out.println("Added "+authors.get(i).getId()+" "+authors.get(j).getId());
				RepastEdge<ISDoS> edge = net.getEdge(authors.get(i), authors.get(j));
				if(edge == null){
					net.addEdge(authors.get(i), authors.get(j));
				} else {
					edge.setWeight(edge.getWeight()+1);
				}
			}
		}
	}
}
\end{lstlisting}

\subsection{Discipline}
\begin{lstlisting}
public class Discipline implements ISDoS{
	private Integer id;

	public Discipline(Integer id) {
		this.id = id;
	}
}
\end{lstlisting}

\subsection{Community}
\begin{lstlisting}
public class Community implements ISDoS {
	private int id;
	private ArrayList<Author> authors = new ArrayList<Author>();
	private ArrayList<Paper> papers = new ArrayList<Paper>();
	
	public Community(){}

	public Community(int id, ArrayList<Author> authors, ArrayList<Paper> papers) {
		this.id = id;
		this.authors = authors;
		this.papers = papers;
	}
}
\end{lstlisting}

\section{Builder and Simulator}
\begin{lstlisting}
public class SocialDynamicsOfScienceBuilder extends DefaultContext<ISDoS>
implements ContextBuilder<ISDoS>  {
	private Double pW;
	private Double pN;
	private Double pD;

	public int getAuthorCount(){
		return getObjects(Author.class).size();
	}

	public int getPaperCount(){
		return getObjects(Paper.class).size();
	}

	public int getDisciplineCount(){
		return getObjects(Discipline.class).size(); 
	}

	public Context<ISDoS> build(Context<ISDoS> context) {

		// Set id for context.
		context.setId("SocialDynamicsOfScience");

		// Build network for context.
		NetworkBuilder<ISDoS> netBuilder = new NetworkBuilder<ISDoS>(
				"coAuthorship network",
				context, false);
		netBuilder.buildNetwork();

		// Get rates for simulation as parameters.
		Parameters params = RunEnvironment.getInstance().getParameters();
		setpW((Double) params.getValue("authors_per_paper"));
		setpD((Double) params.getValue("papers_per_author"));
		setpN((Double) params.getValue("papers_per_discipline"));


		Network<ISDoS> net = (Network<ISDoS>)context.getProjection("coAuthorship network");

		// Initialize a network for simulation.
		// TODO: Get an initial data set from outside from a file.
		Author A = new Author(0);
		Discipline D = new Discipline(0);
		Paper P = new Paper(0);

		// Author our only agent but we need to keep track of other entities.
		context.add(A);
		context.add(D);
		context.add(P);

		net.addEdge(A, A);
		A.getDisciplines().add(D);
		A.getPapers().add(P);
		P.getAuthors().add(A);
		P.setDiscipline(D);

		// Since in this stage, I do not implemented split/merge of disciplines,
		// - I need more than one discipline.

		Author A1 = new Author(1);
		Author A2 = new Author(2);
		Author A3 = new Author(3);
		Author A4 = new Author(4);

		Discipline CS = new Discipline(1);
		Discipline MATH = new Discipline(2);
		Discipline PHY = new Discipline(3);
		Discipline PHIL = new Discipline(4);

		Paper P1 = new Paper(1);
		Paper P2 = new Paper(2);
		Paper P3 = new Paper(3);
		Paper P4 = new Paper(4);

		context.add(A1);
		context.add(A2);
		context.add(A3);
		context.add(A4);

		context.add(P1);
		context.add(P2);
		context.add(P3);
		context.add(P4);

		context.add(CS);
		context.add(MATH);
		context.add(PHY);
		context.add(PHIL);

		net.addEdge(A1, A2);
		net.addEdge(A2, A3);
		net.addEdge(A1, A3);
		A1.getDisciplines().add(CS);
		A1.getPapers().add(P1);
		P1.getAuthors().add(A1);
		A2.getDisciplines().add(CS);
		A2.getPapers().add(P1);
		P1.getAuthors().add(A2);
		A3.getDisciplines().add(CS);
		A3.getPapers().add(P1);
		P1.getAuthors().add(A3);
		P1.setDiscipline(CS);

		net.getEdge(A2, A3).setWeight(2);
		A2.getDisciplines().add(MATH);
		A2.getPapers().add(P2);
		P2.getAuthors().add(A2);
		A3.getDisciplines().add(MATH);
		A3.getPapers().add(P2);
		P2.getAuthors().add(A3);
		P2.setDiscipline(MATH);

		net.addEdge(A3, A3);
		A3.getDisciplines().add(PHY);
		A3.getPapers().add(P3);
		P3.getAuthors().add(A3);
		P3.setDiscipline(PHY);

		net.addEdge(A4, A);
		A4.getDisciplines().add(PHIL);
		A4.getPapers().add(P4);
		P4.getAuthors().add(A4);
		A.getDisciplines().add(PHIL);
		A.getPapers().add(P4);
		P4.getAuthors().add(A);
		P4.setDiscipline(PHIL);

		// An example community.
		Community community = new Community();
		context.add(community);

		SocialDynamicsOfScienceSimulator simulator = new SocialDynamicsOfScienceSimulator(pW, pN, pD);
		context.add(simulator);

		return context;
	}
}

public class SocialDynamicsOfScienceSimulator implements ISDoS {
	private Double pW;
	private Double pN;
	private Double pD;

	public SocialDynamicsOfScienceSimulator(Double pW, Double pN, Double pD) {
		this.pW = pW;
		this.pN = pN;
		this.pD = pD;
	}

	public void select() {
		Context<ISDoS> context = ContextUtils.getContext(this);
		System.out.println("stepper");
		// Randomly get an author from context.
		Iterable<ISDoS> authors = context.getRandomObjects(Author.class , 1); // Since this methods returns iterable object,
		// Selects an author uniformly distributed randomly from context.
		Author author = new Author();
		// An implementation walk around, cused by usage of built-in context.getRandomObjects(Author.class , 1) method.
		for(ISDoS a : authors){
			author = (Author)a;
		}
		
		// Calls biased random walk by passing relavent data.
		Paper paper = new Paper(context.getObjects(Paper.class).size());
		context.add(paper);
		author.step(pW, paper);
		
		// With a probability seed that given as a parameter adds an author network or not.
		if(RandomHelper.nextDoubleFromTo(0, 1) < this.pN){
			Author newAuthor = new Author(context.getObjects(Author.class).size());
			context.add(newAuthor);
			Paper newPaper =  new Paper(context.getObjects(Paper.class).size());
			context.add(newPaper);
			newPaper.getAuthors().add(newAuthor);
			newAuthor.getPapers().add(newPaper);
			
			Iterable<ISDoS> coAuthors = context.getRandomObjects(Author.class , 1);
			// Selects an author uniformly distributed randomly from context.
			Author coAuthor = new Author();
			for(ISDoS a : coAuthors){
				coAuthor = (Author)a;
			}
			coAuthor.step(pW, newPaper);
		}
		if(RandomHelper.nextDoubleFromTo(0, 1) < this.pD){
			evoluateDisciplines();
		}
		
	}
}
\end{lstlisting}

\section{Documentation}
The codes are documented properly, you can see them from the Documentation/index.html file.

\newpage
