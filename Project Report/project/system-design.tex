\chapter{System Design}
\section{Models}
\paragraph{}
The network implemented as three main model; papers, authors, and disciplines.
It represents a social network among authors.
In this simulation the social network starts with one author writing one paper in one discipline.
During the simulation the network evolves as new authors join, new papers are written, and new disciplines emerge over time.
\paragraph{}
Parameters will obtained from data sets to calculate related, portions.
Simulation uses them as probability distribution seeds.
\section{Parameters}
\paragraph{}
In every step the simulation starts with choosing an author with uniformly distributed probability.
Then starts to walk from this initial author, with creating a new paper.
We modelled these behaviors through a biased random walk. 
The length of the random walk determines the number of co-authors of that related paper. 
At each step in the walk, the author visits a node (starting with itself) and decides to stop with probability pW or to continue the search for additional authors with probability 1-pW.
\paragraph{}
At every time step, with probability pN, the simulation also add a new author to the network with a new paper. 
The parameter pN regulates the ratio of papers to authors.
\paragraph{}
We introduce a novel mechanism to model the evolution of disciplines by splitting and merging communities in the social collaboration network.
The idea, motivated by the earlier observations from the APS data, is that the birth or decline of a discipline should correspond to an increase in the modularity of the network.
Two such events may occur at each time step with probability pD.